\section{Обзор литературы}
\label{sec:literature}
В данной курсовой работе будет затрагиваться тема светодиодных экранов и для полного понимания и погружения в специфику, необходимо обозначить ключевые определения и термины.
\subsection{Определения и термины}

\subsubsection{Светодиодные экраны (LED-экраны)}

Светодиодный экран представляет собой тип дисплея, который использует светодиоды (Light Emitting Diodes) для создания изображения. Он обладает высокой яркостью, контрастностью и эффективностью по сравнению с другими видами дисплеев. Светодиодные экраны часто используются в рекламе, информационных табло, телевидении и других областях.

\subsubsection{Контроллер светодиодных экранов}

Контроллер светодиодных экранов - это устройство или программное обеспечение, которое управляет работой светодиодного экрана. Он отвечает за передачу данных и управляющих сигналов на светодиоды, обеспечивая корректное отображение изображения.

\subsubsection{Эмуляция}

Эмуляция - это процесс имитации работы одного устройства или программы с использованием другого устройства или программы. В контексте данной работы, эмуляция контроллера светодиодных экранов означает создание программного или аппаратного средства, способного имитировать функциональность контроллера для целей тестирования и диагностики.

\subsubsection{Диагностика светодиодных экранов}

Диагностика светодиодных экранов представляет собой процесс выявления и анализа неисправностей, аномалий и проблем, возникающих в работе светодиодных экранов. Она включает в себя методы тестирования, мониторинга и анализа, направленные на обеспечение надежной и бесперебойной работы светодиодных экранов.


\subsubsection{Технологии эмуляции и диагностики}

В литературе существует множество технологий и методов, связанных с эмуляцией и диагностикой светодиодных экранов. Это включает в себя разработку специализированных программных средств, аппаратных устройств, а также методики тестирования и анализа работы светодиодных экранов.

\subsubitem{Применение в практике}

В данной главе обзора литературы также рассматривается практическое применение систем эмуляции контроллеров светодиодных экранов для задач диагностики. Это включает в себя области, где такие системы могут быть полезными, например, в рекламной индустрии, на спортивных событиях, в торговых центрах и многих других.