\subsection{Разработка схем алгоритмов}


\subsubsection{Разработка схемы функции \texttt{compress{\textunderscore}file()}}

Функция \texttt{compress{\textunderscore}file()} сжимает выбранный файл и возвращает сообщение о ошибке или об успешной работе функции.



Алгоритм по шагам:
\begin{enumerate}
    \item[1] Начало.
    \item[2] Если файл можно сжать (он не является архивом), то переход к пункту 2, иначе -- к пункту 9.
    \item[3] Прочитать файл, из прочитанных данных создать вектор с частотами символов.
    \item[4] Создать очередь из узлов для будущего бинарного дерева.
    \item[5] Построить из преоритетной очереди дерево Хаффмана.
    \item[6] Создать таблицу кодировки символов используя обход графа в глубину.
    \item[7] Заново пройдясь по файлу и используя таблицу кодировок, преобразовать начальные данные в закодированную последовательность.
    \item[8] Вывести сжатые данные вместе с таблицей частот в новый архивный файл.
    \item[9] Конец.
\end{enumerate}



\subsubsection{Разработка схемы функции \texttt{decompress{\textunderscore}file()}}

Функция \texttt{decompress{\textunderscore}file()} разархивирует выбранный файл и так же возвращает строку с ошибкой или уведомлением об успешной работе функции.



Алгоритм по шагам:
\begin{enumerate}
    \item[1] Начало.
    \item[2] Если файл является архивом, то переход к пункту 2, иначе -- к пункту 8.
    \item[3] Прочитать файл, извлечь таблицу частот символов и сжатые данные.
    \item[4] Создать очередь из узлов для будущего бинарного дерева.
    \item[5] Построить из преоритетной очереди дерево Хаффмана.
    \item[6] Раскодировать сжатые данные в первоначальные.
    \item[7] Вывести исходный набор данных в новый файл.
    \item[8] Конец.
\end{enumerate}