\subsection{Разработка алгоритмов}


Схемы алгоритмов функций \texttt{compress{\textunderscore}file()} и \texttt{decompress{\textunderscore}file()} приведена в приложении Б.



По схеме видно что эти два метода состоят из множества других подфункций.
В каждой из них может произойти ошибка: от невозможности открытия файла, до выхода за пределы вектора или очереди.
Поэтому в таких случаях функция возвращает строку, содержащую информацию об ошибке.
Любое значение, которое возвращают эти две функции выводится на экран в специальном отдельном окне \texttt{MyBox}.



При разработке алгоритма были использованы такие стандартные библиотеки языка C++ как fstream, vector, bitset, queue, memory, algorithm.



С помощью заголовочного файла fstream производится чтение и запись информации в файл. 
Функции \texttt{std::fstream::read} и \texttt{std::fstream::write} дают возможность читать и записывать данные из (в) файл побайтово, регулируя количество выводимых байт.
Считывая и записывая таким образом символы и таблицу частот, у нас уменьшается вероятность получить ошибку при чтении или неправильно записать таблицу в файл.
Чтобы воспользоваться данными функциями чтения и записи нужно в качестве аргументов передавать указатель типа \texttt{char*} и количество байт, используемых в одной операции.
Для преобразования данных в нужный тип в программе задействован оператор \texttt{reinterpret{\textunderscore}cast<char*> (\&value)}, где \texttt{\&value} это адрес данных в памяти компьютера.



В большинстве случаев вся длина конечной закодированной последовательности не будет кратна размеру одного байта.
Поэтому, при записи данных в файл после таблицы частот записывается длина конечной последовательности в байтах и остаток из бит не образующих один байт.



Для записи в файл исходной последовательности используется структура \texttt{std::bitset<CHAR{\textunderscore}BIT>}.
В ее конструктор передается код символа и с его помощью эта кодировка преобразуется в набор бит(каждый символ '1' или '0' превращается в соответствующий бит).
Потом структура содержащая 8 бит приводится к типу \texttt{unsigned long} и с помощью оператора \texttt{static{\textunderscore}cast<unsigned char>} преобразуется в беззнаковый символный тип.
Далее такая операция проводится в цикле до тех пор пока мы не запишем всю последовательность вместе с битовым остатком.



При чтении сжатых данных производится похожая манипуляция.
Главное отличие в том, что при чтении сначала \texttt{unsigned char} образовывает структуру битового набора.
Потом с помощью функции \texttt{std::bitset::to{\textunderscore}string()} структура приводится к типу \texttt{std::string}.
Далее разархивация происходит по алгоритму описанному ранее.

