\subsection{Выбор среды разработки}

Выбор среды разработки является важным этапом разработки любого программного обеспечения, в том числе и для микроконтроллеров STM32. Среда разработки предоставляет набор инструментов, необходимых для создания, компиляции, отладки и тестирования программного кода.

На выбор среды разработки влияют следующие факторы:
\begin{itemize}
    \item Язык программирования: среда разработки должна поддерживать язык программирования, на котором будет написан программный код. Для микроконтроллеров STM32 наиболее популярными языками программирования являются C и C++.
    \item Функциональность: среда разработки должна предоставлять необходимый набор функций для разработки программного кода для микроконтроллеров STM32. К таким функциям относятся:
    \begin{itemize}
        \item Поддержка различных микроконтроллеров STM32
        \item Поддержка различных отладчиков
        \item Поддержка различных библиотек
        \item Поддержка различных средств тестирования
    \end{itemize}
    \item Цена: среды разработки бывают платными и бесплатными. Платные среды разработки, как правило, предоставляют более широкий набор функций и возможностей, чем бесплатные.
\end{itemize}

\subsubsection{STM32CubeIDE}
\textit{STM32 CubeIDE}, среда разработки под контроллеры \textit{STM32} от компании производителя чипов \textit{STMicroelectronics}. Данный программный продукт поставляется совершенно бесплатно. Что является большим плюсом.

На время написания статьи программу можно скачать, зарегистрировав аккаунт и указав европейскую страну. Так же необходимо использовать ВПН. Интерфейс \textit{CubeIDE} интуитивно понятен и содержит всё, что нужно для комфортной разработки (отладчик с отладкой в real time, просмотр переменных и т.д.).

Так же стоит отметить, что \textit{CubeIDE} все время развивается и поддерживается разработчиком. К минусам отнесем отсутствие документации на русском языке.

\subsubsection{\textit{Keil}}
\textit{Keil} — одна из самых мощных IDE для разработки программ под микроконтроллеры STM32. \textit{Keil} имеет свой собственный компилятор, позволяющий комфортно отлаживать программный код. В IDE используется язык программирования С\\С++. Так же \textit{Keil} имеет в своем распоряжение симулятор, который позволяет эмулировать некоторое железо, например UART.

Минусом данной IDE является платная лицензия. А так же то, что программа работает только в ОС Windows. На просторах интернета можно скачать активатор для \textit{Keil}, но в таком случае у пользоваВходе телей наблюдаются вылеты и подвисания программы.

\subsubsection{IAR Embedded Workbench for ARM (IAR-EWARM)}
Еще один хороший редактор кода с компилятором C\\C++ для микроконтроллеров STM32. Этот редактор кода имеет в своем распоряжении более 4000 примеров по работе с периферией STM32. Так же IDE IAR в автоматическом режиме проверяет ваш код на правила MISRA C (MISRA C: 2004). Это правила написания отказа устойчивого программного кода.

Редактор поддерживает все контроллеры STM32, библиотеки для работы с периферией, а так же плагин для работы с RTOS (операционная система реального времени).

Минусом использования IDE IAR это то, что она является платной.

\subsubsection{CodeGrip}
Еще один редактор кода для микроконтроллеров STM32 от компании MICROE. Одним из отличий данного редактора является то, что редактор имеет такие компиляторы как: microC, microPascal, microBasic. Компиляторы разработаны под экосистему Microe, оптимизированы под отладочные комплекты компании. Большим минусом является платная лицензия на компилятор.

В заключение необходимо сказать о выборе STM32 IDE. Каждый из редакторов имеет и плюсы и минусы, в частности я выбрал для себя CubeIDE в связке с CUBEMX. Данная среда разработки удовлетворяет всем моим запросам, а так же что немало важно она является бесплатной.

\subsubsection{Итоговый выбор}

В данной курсовой работе будет использоваться среда разработки STM32CubeIDE. Эта среда разработки является бесплатной и поддерживает широкий набор микроконтроллеров STM32. Кроме того, она имеет простой и интуитивно понятный интерфейс, что делает её удобной для начинающих разработчиков.