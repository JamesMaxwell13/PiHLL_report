\subsection{Конфигурация микроконтроллера}
Конфигурация микроконтроллера является важным этапом его подготовки к работе. Для корректной работы микроконтроллера необходимо произвести настройку периферии. К ней относится:
\begin{itemize}
    \item Источник тактирования
    \item Делители частоты
    \item Порта ввода-вывода общего назначения
    \item Интерфейс отладки
    \item Дополнительные интерфейсы
\end{itemize}
Для конфигурации микроконтроллера STM32 будет использоваться среда разработки STM32CubeMX. Эта среда разработки является бесплатной и поддерживает широкий набор микроконтроллеров STM32. Кроме того, она имеет простой и интуитивно понятный интерфейс, что делает её удобной для начинающих разработчиков.

Использование STM32CubeMX для конфигурации микроконтроллера STM32 позволяет значительно упростить и ускорить этот процесс. Среда разработки предоставляет графический интерфейс, который позволяет легко и быстро настроить все необходимые параметры периферии.

\subsubsection{Конфигурация тактирования}
\subsubsection{Конфигурация портов ввода-вывода}
\subsubsection{Конфигурация интерфейса отладки}
\subsubsection{Конфигурация дополнительных интерфейсов}