\section{Постановка задачи}
\label{sec:task}

Для реализации программы необходимо детально изучить алгоритмы считывания данных и построения деревьев поиска для дальнейшей компоновки кодов символов.
При написаннии рабочего кода программы можно столкнуться с трудностями при чтении информации в файл, выводе таблицы кодировки и самих закодированный данных в архивный файл.
Также не стоит забывать про огранниченные возможности персональных компьютеров и правильно организовывать запоминание информации в промежуточные сущности и контейнеры.



В данном приложении необходимо реализовать просмотр всех возможных файлов для сжатия и разархивации. 
Предусмотреть возможность выбора пользователем уже сжатого файла для дальнейшей архивации (программа может повести себя непредсказуемо).
При любой операции программа должна предупреждать пользователя о результатах данной операции.
Стоит предусмотреть возможность изменения названия файла при разархивации и конфликте имен.



Основная часть алгоритма для сжатия должна разделяться на следующие логические части:
\begin{itemize}
    \item cчитывание исходного файла;
    \item построение дерева поиска;
    \item построение таблицы шифрования данных;
    \item шифрование данных;
    \item запись таблицы кодов в архивный файл;
    \item запись закодированных данных в архивный файл.
\end{itemize}



Для алгоритма разархивации необходимо иметь такие методы:

\begin{itemize}
    \item считывание исходного файла;
    \item считывание таблицы кодирования данных;
    \item дешифрование данных;
    \item запись декодированных данных в новый файл.
\end{itemize}
