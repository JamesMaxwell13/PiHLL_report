\section{Разработка функциональной схемы}
\label{sec:func}

\subsection{Блок питания}
Блок питания обеспечивает питанием все элементы устройства. Он состоит из следующих элементов:
\begin{itemize}
    \item Схема защиты от смены полярности предотвращает выход из строя устройства при подключении питания с неправильной полярностью. Она включает в себя диод шотки, конденсатор номиналом 100nF, резистор 100k и PMOSFET AO3401A.
    \item Схема фильтрации от помех подавляет помехи, которые могут поступать в устройство с питающей сетью. Она реализована за счет использования нескольких фильтрующих керамических конденсаторов номиналом 100nF.
    \item Преобразователь входного напряжения преобразует входное напряжение 5 В в 3.3 В, необходимое для питания микроконтроллера и других элементов устройства. Он реализован с помощью микросхемы AMS11173.3.
    \item Индикация наличия питания показывает, что устройство подключено к питанию.
\end{itemize}

\subsection{Блок микроконтроллера}
Блок микроконтроллера содержит микроконтроллер STM32F103RCT6 и необходимые для его работы элементы. Он состоит из следующих элементов:
\begin{itemize}
    \item Тактирование осуществляется внешним кварцевым резонатором NX2520SA.
    \item Отладка осуществляется с помощью выводов SWCLK и SWDIO микроконтроллера.
    \item Коммуникации с внешними устройствами осуществляются с помощью выводов USART и I2C микроконтроллера.
\end{itemize}

\subsection{Блок преобразования уровней} 

Блок преобразования уровней предназначен для согласования уровней сигналов, работающих на различных уровнях напряжения. Он состоит из двух преобразователей уровней TXS0108EPWR и TXS0104EPWR. Каждый преобразователь имеет восемь или четыре канала, которые могут преобразовывать сигналы с уровня 3.3 В на уровень 5 В или наоборот.
