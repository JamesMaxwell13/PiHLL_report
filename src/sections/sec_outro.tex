\sectionCenteredToc{Заключение}
\label{sec:outro}

В ходе курсового проектирования были раcсмотрены и изучены основные алгоритмы сжатия файлов.
Наиболее детально был изучен алгоритм кодирования Хаффмана.
Для реализации данного алгоритма использованы встроенные функции и структуры даных языка C++.
В ходе работы закреплены и использованы на практике знания полученные при изучении дисциплины «Программирование на языках высокого уровня».



Для визуализации и создания полноценного оконного приложения использованы библиотеки и основные виджеты фреймворка QT.
Были изучены основные правила взаимодействия графического интерфейса между собой и реализованы в курсовом проекте.



На основе полученных знаний можно углубиться в проектирование приложений похожей тематики.
Для улучшения качества следующих проектов необходимо более досконально изучить математичекий аппарат описывающий сжатие информации и возможности графического интерфейса библиотеки QT.



Данная курсовая работа может послужить хорошей основой для углубленного изучения языка и фреймворков C++.
Знания, полученные в процессе написания работы пригодятся в ходе дальнейшего обучения на специальности и помогут реализовать себя в будущей карьере инженера-системотехника.