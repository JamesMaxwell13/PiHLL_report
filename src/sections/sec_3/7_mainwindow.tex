\subsubsection{Класс главного окна приложения \texttt{MainWindow}}

Данный класс как и несколько предыдущих наследуется от уже определенного в фреймворке QT класса \texttt{QMainWindow}\cite{QT}.



Так как с этим классом кроме функции \texttt{int main()} ничего не взаимодействует, то соответственно и все методы и поля (кроме конструктора и деструктора) имеют приватный спецификатор доступа.
Детально рассмотрим поля и методы данного класса.



\texttt{Ui::MainWindow *u}\cite{QT}.
Представляет переменную описывающую файл mainwindow.ui с помощью которого описывается весь интерфейс и взаимодействие его элементов друг с другом.



\texttt{QWidget *widget}\cite{QT}.
Виджет предназначен для компоновки других графических элементов: кнопок, виджетов \texttt{MyTreeView} и \texttt{MyListView}.



\texttt{QPushButton *compress} и \texttt{QPushButton *decompress}\cite{QT}.
Поля кнопок, отвечающих за архивацию и разархивацию выбранных файлов.



\texttt{MyTreeView *treeview} и \texttt{MyListView *listview}.
Поля для вывода файловой информации (см. описание соответствующих классов).



\texttt{QHBoxLayout *views}, \texttt{QHBoxLayout *buttons} и \texttt{QVBoxLayout *vlayout}\cite{QT}.
Эти объекты автоматически выравнивают все виджеты, добавленные в них. 
Для кнопок и и файловых виджетов предназначены два разных горизонтальных объекта выравнивания.
Все вместе эти два объекта состоят в единов вертикальном классе выравнивания.



\texttt{QTimer *timer}\cite{QT}.
Специальный таймер, предопределеный в фреймворке QT.  
С его помощью регулируется время стостояния нажатых кнопок.



\texttt{void working{\textunderscore}file(const QModelIndex \&index)}.
Данный метод предназначен для непосредственной работы с выбранным файлом: его можно открыть, заархивировать или развархивировать.



\texttt{void on{\textunderscore}treeView{\textunderscore}clicked(const QModelIndex \&index)}
Эта функция вызывается при нажатии элемент на виджет вывода директорий \texttt{MyTreeView}.
Она меняет директорию виджета вывода файлов \texttt{MyListView} на выбранную и выводит все файлы, находящиеся в ней.



\texttt{void on{\textunderscore}listView{\textunderscore}clicked(const QModelIndex \&index)}
Здесь обрабатывается двойное нажатие на элемент виджета вывода файлов.
Когда на определенный файл два раза кликают мышью, он открывается в соответствующем редакторе, установленном по умолчанию для открытия данного типа файлов.

