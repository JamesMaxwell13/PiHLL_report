\subsubsection{Класс шаблонной очереди \texttt{Queue{\textunderscore}t} и компаратора \texttt{LowestPriority}}


Класс пользовательской очереди \texttt{Queue{\textunderscore}t} предназначен для начального хранения данных.
В контексте данной курсовой работы этот класс используется при первом проходе по сжимаемому файлу.
Данная структура данных создана на основе шаблона \texttt{std::priority{\textunderscore}queue}\cite{queue}.
Для шаблона установлен хранимый тип \texttt{Node::ptr}, используемый в основе контейнер \texttt{std::vector<Node::ptr>} и класс компратор \texttt{LowestPriority}\cite{queue}.
Все эти параметры позволяют создать очередь из узлов \texttt{Node}, в которой все новые значения записываются в приоритетном порядке с наименьшим элементом в верши не очереди.



\texttt{Std::priority{\textunderscore}queue<Node::ptr, std::vector<Node::ptr>, LowestPriority> q{\textunderscore}data}.
Это главное и единственное поле содержит саму очередь с данными.



\texttt{Queue{\textunderscore}t(std::vector<int>\& frequency)}.
Конструктор создает очередь из вектора содержащего частоты символов.
Проходя по вектору, в конструкторе создается новый узел и записывается в очередь пока вектор не закончится.



\texttt{Void build{\textunderscore}tree()}.
Этот метод на основе уже заполненной очереди строит бинарное дерево по алгоритму, описанному в предыдущем разделе.



\texttt{Node::ptr top()}.
Функция возвращает верхний узел в очереди. 
Именно он будет вершиной дерева Хаффмана.