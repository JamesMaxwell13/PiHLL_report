\subsubsection{Класс окна уведомлений \texttt{MyBox}}

Класс \texttt{MyBox} предназначен для вывода на экран сообщения об успешности проведения операций архивирования и разархивирования файлов.
Он наследуется от встроенного в QT виджета \texttt{QMessageBox}\cite{qt_doc}.



Для создания графического интерфейса необходимо лишь изменить конструктор.
В качестве аргумента \texttt{MyBox(QString message)} принимает строку, которая будет выводиться на виджете \texttt{QMessageBox}.



\subsubsection{Класс виджета вывода директорий \texttt{MyTreeView}}

Элемент \texttt{MyTreeView} выводит на экран файловую систему \texttt{QFileSystemModel}\cite{qt_doc} определенную в фреймворке QT.
Он представляет все файлы и директории в виде иерархического дерева.
При таком выводе файловой модели на экран внутренние файлы и дочерние папки выпадают списком из родительской директории.



Данный класс наследуется от уже предопределенного в QT класса \texttt{QTreeView}\cite{qt_doc}.
Рассмотрим поля класса \texttt{MyTreeView}.



\texttt{QFileSystemModel *dirmodel} представляет сам объект файловой системы\cite{QT}.



\texttt{QString sPath}\cite{QT}. 
Данная переменная содержит путь к рабочей папке, где программа будет архивировать и разархивировать файлы.



\texttt{Int nameWidth}.
Это поле содержит длину столбца "Name" в виджете \texttt{MyTreeView}.
По умолчанию, чтобы виджет показывал полностью название большинства папок и директорий, его значение установлено как 240.



Так же как и в предыдущем классе, для функционирования интерфейса необходим только конструктор \texttt{MyTreeView(QString path)}.
В качестве аргумента в него поступает строка, которая содержит начальный путь к папке, которую будет показывать виджет.
По умолчанию этот путь нулевой, а значит \texttt{MyTreeView} выведет на экран дисковое пространство компьютера\cite{qt_doc}.
Также в конструкторе устанавливаются остальные параметры виджета: ширина столбцов, параметры вывода элементов \texttt{QFileSystemModel} и другое.



\subsubsection{Класс виджета вывода файлов \texttt{MyListView}}

\texttt{MyListView} очень похож на класс \texttt{MyTreeView}.
Он также наследуется от встроенного виджета \texttt{QListView}\cite{qt_doc}.
Но все же есть и различия.
В отличие от \texttt{MyTreeView}, данный класс выводит только файлы, но не в виде дерева, а в виде простейшего списка.



\texttt{QString sPath} тоже содержит в себе начальный путь к папке, которая откроется при открытии оконного приложения.



\texttt{QFileSystemModel *filemodel} представляет объект файловой системы, которая содержит в себе только файлы.



Конструктор \texttt{MyListView(QString path)} точно так же как и в предыдущем классе устанавливает в качестве объектной модели\cite{qt_doc}\texttt{filemodel} и начальный путь к папке.
В отличие от \texttt{MyTreeView} в данном классе нет определеных столбцов с информацией (выводятся только названия файлов).
