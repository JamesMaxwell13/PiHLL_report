\subsubsection{Класс кода Хаффмана для файла \texttt{Huffman{\textunderscore}code}}


Класс \texttt{Huffman{\textunderscore}code} хранит в себе имя файла, данные для записи в другой файл и функции обрабатывающие эти данные.
Рассмотрим его поля и методы.



\texttt{Std::string filename}.
Поле содержит имя файла, выбранного для архивации или разархивирования.



Для части нижеперечисленных методов установлен спецификатор доступа \texttt{private} с целью предотвратить использование данных функций вне публичных методов.

 

\texttt{Std::string message}.
Содержит в себе последовательность данных.
В случае сжатия файла представляет собой конечные закодированные данные.
Если архив преобразуется в первоначальный файл, то \texttt{message} хранит исходные раскодированные данные.



\texttt{Int read{\textunderscore}file(std::vector<int>\& frequency)}.
Данный метод считывает исходный файл и заполняет частотную таблицу в виде вектора.
В векторе под индексом, равным ASCII коду символа хранится количество повторений данного символа в кодируемой последовательности данных.



\texttt{Void make{\textunderscore}code(Node::ptr\& node, std::string str, std::vector<std::string>\& codes)}.
Этот метод рекурсивно формирует кодировку каждого символа для исходного алфавита последовательности из бинарного дерева поиска.
По своей сути он напоминает обход графа в глубину.
В ходе прохода по дереву кодировка формируется в переменной \texttt{str} и при достижении листа записывается в вектор \texttt{codes}.



\texttt{Int message{\textunderscore}to{\textunderscore}code(const std::vector<std::string>\& codes)}.
Функция преобразует первоначальный набор данных в закодированную последовательность, которая сохраняется в переменную \texttt{message}.



\texttt{Int write{\textunderscore}file(std::vector<int>\& frequency)}.
Метод записывает таблицу частот и всю закодированную информацию в архивный файл.



\texttt{Int read{\textunderscore}decoding{\textunderscore}file(std::vector<int>\& frequency)}.
Данная функция считывает из файла частотную таблицу и всю закодированную информацию после нее.



\texttt{Void make{\textunderscore}char(const Node::ptr\& root, std::string\& text)}.
Этот метод по дереву Хаффмана преобразует закодированные данные в проследовательность данных до сжатия.



\texttt{Int write{\textunderscore}decoding{\textunderscore}file(std::string\& text)}.
Метод создает файл и записывает в него всю информацию, которая была сжата в архиве.



Остальные методы класса \texttt{Huffman{\textunderscore}code} являются публичными и бутуд использоваться напрямую в самом оконном приложении.



\texttt{Huffman{\textunderscore}code(std::string file)} и \texttt{\~Huffman{\textunderscore}code()}. 
Конструктор и деструктор предназначен для установки поля \texttt{filename} и очистки полей \texttt{filename} и \texttt{message} с соответственно.
Прямого влияния на алгоритм сжатия они не оказывают.



\texttt{Bool is{\textunderscore}compressed()}. 
Метод возвращает \texttt{true}, если выбранный файл является уже сжатым, и \texttt{false}, если файл не являетсчя архивом.



\texttt{Bool can{\textunderscore}compress()}.
Метод возвращает \texttt{true}, если выбранный файл можно сжать, и \texttt{false}, если файл уже сжат.



\texttt{Std::string compress{\textunderscore}file()}.
Данная функция собственно сжимает файл. 
Она использует некоторые вышеперечисленные приватные методы, а также методы класса \texttt{Queue{\textunderscore}t}.
В качестве возвращаемого параметра, функция передает сообщение об успешности или ошибки проведения операции сжатия файла.



\texttt{Std::string decompress{\textunderscore}file()}.
В отличие от предыдущего метода, этот разархивирует файл, используя оставшиеся защищенные методы.
Функция возвращает сообщение об ошибках или успешном выполнении.