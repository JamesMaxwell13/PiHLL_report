\subsection{Структура входных и выходных данных}

В качестве входных данных могут использоваться любые файлы приводимые к текстовому типу: .txt, .tex, .html, .cpp, .odt и так далее. 
Программа работает с файловой системой NTFS и операционной системой Windows 10.



При сжатии создается новый файл, к названию которого после расширения добавляется постфикс ".arch".
Чтобы файл не занимал много места в памяти, раширение не записывается в сам архивный файл, а остается в его названии.
Таким образом файл "index.html" архивируется в файл "index.html.arch".



При разархивации файла постфикс ".arch" убирается. 
Если файл с таким же названием уже существует (без постфикса), то перед расширением вставляется подстрока "\textunderscore1".



При проведении каждой операции выводится уведомление об ошибке или ее успешном завершении.