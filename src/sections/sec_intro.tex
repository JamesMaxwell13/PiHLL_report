\sectionCenteredToc{Введение}
\label{sec:intro}

Семимильными шагами ступает прогресс по планете Земля. Каждый день мы можем наблюдать,
как в мире происходит так называемая «цифровая революция», которая
началась еще в последних десятилетиях прошлого века. Связана она с распространением
информационных технологий и проникновением их во все сферы жизни общества.



В современном мире технологий, многие компьютерные программы и файлы занимают на электронных накопителях информации все больше и больше места. 
Общество со временем начало использовать специальные программы для сжатия информации и последющего извлечения -- архиваторы. 
Со временем, такое программное обеспечение стало неотъемлимой частью нашей работы за компьютером. 
Но, казалось бы, простое сжатие информации тоже имеет свои загвоздки и ньюансы.
Поэтому, пользователям нужно понимать основные механизмы архивации данных.



Для изучения и создания программы архивации файлов в данном курсовом проекте будет использоваться компилируемый строго типизированный язык программирования С++, а также кроссплатформенный фреймворк QT.
Чтобы увеличить эффективность программы и правильно структурировать иходный код, для реализации основного алгоритма используется объектно-ориентированная парадигма программирования.



С помощью абстракции, инкапсуляции и полиморфизма\cite{lafore} структурируется и упрощается основной алгоритм программы. 
Стандартная библиотека шаблонов\cite{lucik} (Standart Template Library, STL) предоставляет обширный набор обобщенных контейнеров и множество методов их обработки. 
Благодаря STL сохранность данных увеличивается, уменьшается вероятность утечек памяти -- код становится более надежным.
Наследование помогает правильно выстроить интерфейс и видоизменить в приложении объекты, предопределенные в фреймворке QT\cite{qt_doc}. 
QT в свою очередь дает возможность эффективно описывать файловую систему\cite{filemodel}, создавать лаконичное визуальное оформление различных виджетов.



Оконное приложение, разработанное в данной курсовой работе, позволяет визуализировать объекты файловой системы используя основные элементы оконных приложений, а также наглядно иллюстрирует работу алгоритма Хаффмана со всеми его недостатками. 